\documentclass[dvips,12pt]{article}

% Any percent sign marks a comment

% Every latex document starts with a documentclass declaration like this
% The option dvips allows for graphics, 12pt is the font size, and article
%   is the style

\usepackage[pdftex]{graphicx}
\usepackage{url}

% These are additional packages for "pdflatex", graphics, and to include
% hyperlinks inside a document.

\setlength{\oddsidemargin}{0.25in}
\setlength{\textwidth}{6.5in}
\setlength{\topmargin}{0in}
\setlength{\textheight}{8.5in}

% These force using more of the margins that is the default style

\begin{document}

% Everything after this becomes content
% Replace the text between curly brackets with your own

\title{MLCD Title}
\author{Katie Henry, Adi Renduchintala, David Snyder}
\date{\today}

% You can leave out "date" and it will be added automatically for today
% You can change the "\today" date to any text you like


\maketitle

% This command causes the title to be created in the document

\section{Goals}
This project is motivated by the task of early prediction of septic shock in ICU patients, using multiples streams of physiologic data. 
We will investigate this problem through the development of three interacting components:
1) Create a Deep Neural Network (DNN) famework to provide probability estimates of whether a patient develops septic shock (David Snyder),
2) Uncover features that explain phenomena observed in our data using Topic Modeling (Adi Renduchintala) and
3) Derive an appropriate policy for optimal decision making using the above components (Katie Henry).


\section{Data}

\section{Approach}

\subsection{Classification with Deep Neural Network}

Recent advances in DNNs have shown state of the art performance in multiple domains, including speech recognition (citations needed). We believe that there are similarities between the physiologic streams of the ICU patients and acoustic data, which we hope to exploit for this project. This component will focus on the development of the DNN architecture and a subsequent study of classification performance under various network topologies and activation functions. 
Eventually, we hope to explore the interaction between the DNN framework and the other components of this project. 
For instance, we could study the effects of augmenting or replacing the input stream with the features generated by topic modeling.


\subsection{Feature Discovery with Topic Modeling}

Although sceptic shock is often labeled as a single condition it is known (citations needed) that there are many variations. 
Furthermore, each variation is preceded by a mixture of different events which lead to the onset of the condition. Because of the variation in underlying structure
of this problem we believe that topic modeling may result in discovering features which reflect this variation and result in more explanatory features. Topic models have been used extensively for physiological signals in the time domain (citation). Typically these features have been generated from a single time series. In our data set we have access to multiple channels of data for each patient. We would like to extend our feature generation to include multiple channels and interaction between channels. 

\subsection{Optimal Decision Making with Katie's stuff}
\begin{thebibliography}{99}
\bibitem{saria2010} S. Saria, 
A. Rajani, 
J. Gould, 
D. Koller, 
A. Penn. 
{Integration of Early Physiological Responses Predicts Later Illness Severity in Preterm Infants},
Science Translational Medicine, Vol. 2, Issue 48 (2010).

\end{thebibliography}



\end{document}
