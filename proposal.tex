\documentclass[dvips,12pt]{article}

% Any percent sign marks a comment

% Every latex document starts with a documentclass declaration like this
% The option dvips allows for graphics, 12pt is the font size, and article
%   is the style

\usepackage[pdftex]{graphicx}
\usepackage{url}

% These are additional packages for "pdflatex", graphics, and to include
% hyperlinks inside a document.

\setlength{\oddsidemargin}{0.25in}
\setlength{\textwidth}{6.5in}
\setlength{\topmargin}{0in}
\setlength{\textheight}{8.5in}

% These force using more of the margins that is the default style

\begin{document}

% Everything after this becomes content
% Replace the text between curly brackets with your own

\title{MLCD Title}
\author{Katie Henry, Adi Renduchintala, David Snyder}
\date{\today}

% You can leave out "date" and it will be added automatically for today
% You can change the "\today" date to any text you like


\maketitle

% This command causes the title to be created in the document

\section{Goals}
This project is motivated by the task of early prediction of septic shock in ICU patients, using multiple streams of continuous physiological data. 
We will investigate this problem through the development of three interacting components:
1) Create a Deep Neural Network (DNN) famework to provide probability estimates of whether a patient develops septic shock (David Snyder),
2) Uncover features that explain phenomena observed in our data using Topic Modeling (Adi Renduchintala) and
3) Derive an optimal decision policy using the above components that is aware of when making a prediction is more costly than delaying prediction until more data has been observed (Katie Henry).


\section{Data}
MIMIC II Waveform Database is a collection of continuous physiological waveforms from patients in the ICU. This includes ECG, blood pressure, oxygen saturation, and pulse readings measured at 125 Hz. Using the subset of these waveforms that have been associated with electronical medical records, we are able to identify patients who have septic shock and the time of onset of septic shock. In all there are 148 waveform records and 3722 hours of data that cover up to three days preceding the onset of septic shock. On average the records contain information about twenty-five hours prior to septic shock onset and the median record has 20 hours of data prior to onset. There are additional records for patients with SIRS, severe sepsis, and no sepsis event.

\section{Approach}

\subsection{Classification with Deep Neural Network}

Recent advances in DNNs have shown state of the art performance in multiple domains, including speech recognition \cite{hinton2012deep}. We believe that there are similarities between the physiologic streams of the ICU patients and acoustic data, which we hope to exploit for this project. This component will focus on the development of the DNN architecture and a subsequent study of classification performance under various network topologies and activation functions. 
Eventually, we hope to explore the interaction between the DNN framework and the other components of this project. 
For instance, we could study the effects of augmenting or replacing the input stream with the features generated by topic modeling.


\subsection{Feature Discovery with Topic Modeling}

Although sceptic shock is often labeled as a single condition it is known (citations needed) that there are many variations. 
Furthermore, each variation is preceded by a mixture of different events which lead to the onset of the condition. Because of the variation in underlying structure
of this problem we believe that topic modeling may result in discovering features which reflect this variation and result in more explanatory features. Topic models have been used extensively for physiological signals in the time domain (citation). Typically these features have been generated from a single time series. In our data set we have access to multiple channels of data for each patient. We would like to extend our feature generation to include multiple channels and interaction between channels. 

\subsection{Towards Reliable Prediction}
In a clinical setting, it can be more harmful for a system to make an incorrect prediction than to make any prediction at all and a clinician might want to have more fine-grainied control of the costs of false negatives and false positives for different disease states. In order to incorporate these concerns, we propose to derive an optimal decision policy over the time sequence that can either predict a disease class (e.g.\ septic shock or not septic shock) or can refuse to make any prediction and that allows the clinician to specify arbitrary costs of different decisions. By incorporating the notion of refusing to predict, we guarantee that a predicted class is not only the prediction with the lowest cost, but also that it's cost is below a certain threshold. That threshold evolves as a function of the likelihood of the prediction given the past data and decisions and the costs at the current time.

\section{Previous Work}
Previous work has shown that measurements derived from continuous physiological data are useful in disease prediction and risk stratification tasks. Clinical studies have shown that in healthy patients heart rates have a multifractal pattern over time and that there is a loss of complexity in the heart rate signal in patients with heart disease (cite Goldberger). Previous reserearch has used features like heart rate variability, sample asymmetry, detrended fluctuation analysis, and approximate entropy to caputure this loss in complexity and to characterize the waveform in addition to other features motivated by clinical knowledge and observations (cite Moorman and Clifford).  There has also been some work on automatically learning repeating motifs and shape signatures in  the waveform and using these to classify patients' health status (cite Seyed and Saria); however, these approaches still focus on each signal in isolation.

\bibliographystyle{plain}

\bibliography{dnn}




\end{document}
